\documentclass[12pt,a4paper]{report}
\usepackage[utf8]{inputenc}
\usepackage[T1]{fontenc}
\usepackage[french]{babel}
\usepackage{hyperref}
\usepackage{geometry}
\geometry{margin=2.5cm}

\title{PhishTracer : Détection et Traçage de Phishing}
\author{[Votre Nom]}
\date{\today}

\begin{document}

\maketitle

\begin{abstract}
PhishTracer est une application web permettant de détecter et tracer les tentatives de phishing à partir d'URL ou d'e-mails suspects. Ce document présente l'architecture, les fonctionnalités et l'utilisation de la solution.
\end{abstract}

\tableofcontents

\chapter{Introduction}
Le phishing est une menace majeure pour la sécurité des utilisateurs sur Internet. PhishTracer vise à fournir un outil simple et efficace pour analyser des liens ou des e-mails suspects et alerter l'utilisateur.

\chapter{Architecture du projet}
L'application est composée de deux parties principales :
\begin{itemize}
  \item \textbf{Frontend} : développé en React, il offre une interface utilisateur moderne et intuitive.
  \item \textbf{Backend} : basé sur Node.js, Express et Firebase, il gère l'authentification, le stockage des données et la logique métier.
\end{itemize}

\section{Schéma global}
\begin{itemize}
  \item L'utilisateur interagit avec l'interface React.
  \item Les requêtes sont envoyées au backend Express.
  \item Les données sont stockées et sécurisées via Firebase (Firestore et Auth).
\end{itemize}

\chapter{Fonctionnalités principales}
\begin{itemize}
  \item Analyse d'URL et d'e-mails pour détecter le phishing
  \item Tableau de bord utilisateur et administrateur
  \item Authentification sécurisée (utilisateurs et admins)
  \item Historique des scans
  \item Notifications de sécurité
\end{itemize}

\chapter{Technologies utilisées}
\begin{itemize}
  \item \textbf{Frontend} : React, TailwindCSS, Axios, Framer Motion
  \item \textbf{Backend} : Node.js, Express, Firebase (Firestore, Auth)
  \item \textbf{Sécurité} : Helmet, Sanitize-html, JWT
\end{itemize}

\chapter{Installation}
\section{Prérequis}
\begin{itemize}
  \item Node.js >= 16
  \item npm
\end{itemize}

\section{Installation du frontend}
\begin{verbatim}
cd PhishTracer-1.0
npm install
npm run dev
\end{verbatim}

\section{Installation du backend}
\begin{verbatim}
cd backend
npm install
npm run dev
\end{verbatim}

\chapter{Utilisation}
\begin{itemize}
  \item Accédez à l'interface web sur http://localhost:5173
  \item Inscrivez-vous ou connectez-vous
  \item Utilisez le scanner d'URL ou d'e-mails
  \item Consultez l'historique et les notifications
\end{itemize}

\chapter{Conclusion}
PhishTracer propose une solution moderne et sécurisée pour lutter contre le phishing. Son architecture modulaire permet une évolution facile et une adaptation à de nouveaux besoins.

\end{document} 